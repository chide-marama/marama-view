\documentclass[10pt]{extarticle} % article doctype, possible font size range from 8pt to 20pt with not all being avaiable.
\setcounter{secnumdepth}{4} % Max. section depth, e.g. 3 means 4.1.1 is possible but not 4.1.1.2
\title{\huge Technical Design Document MaramaUU}
\author{Erik van Lune, Tim Hintzbergen, Wilbert Schepenaar, <nextname>    \\s1095043, s1097561, s1094484, <nextcode>
\\\\ICTGPb
\\Wilco Moerman}
\date{May 8th, 2018 <TODO>}

% include packages
\usepackage{graphicx}
\usepackage{caption}
\usepackage{mathabx}
\usepackage{wrapfig}
\usepackage[margin=1.0in]{geometry} % Sets page margin, 2.0in is default.
\usepackage{titlesec}
\usepackage{hyperref}

% custom commands
\newcommand{\myparagraph}[1]{\paragraph{#1}\mbox{}\\} % Without \mbox{} all newlines will be ignored, making the first sentence appear on the same line as a paragraph title.
\def\code#1{\texttt{#1}}

\DeclareCaptionFormat{cancaption}{#1#2#3\par} % Normal format actually
\DeclareCaptionLabelFormat{cancaptionlabel}{#1}
\captionsetup[figure][number]{format=cancaption,labelformat=cancaptionlabel}
\graphicspath { {images/} }
\begin{document}
    \maketitle
    \thispagestyle{empty}
    \newpage
    %------------------------------------------------------------------------------------------------------------------------------------------------------- Introduction
    \newpage
    \setcounter{page}{1}
    \section {Introduction}
    This document contains all technical aspects of the product.
    It describes the requirements, the architecture, the design choices and the epics.
    The requirements talk about all functional and non-functional requirements of the product.
    The architecture explains the form of the product, offers a clear overview and talks about some minute details.
    The list of design choices illustrates and defend the various choices that have been made during the design and construction of the product.
    Finally the epics list all coherent components of the product and their submodules
    \newpage

    \tableofcontents{}
    \newpage

    %------------------------------------------------------------------------------------------------------------------------------------------------------- Requirements
    \section{Requirements}
    Functional and Non-functional requirements.
    \newpage

    %------------------------------------------------------------------------------------------------------------------------------------------------------- Architecture
    \section{Architecture}
    This section describes the architecture of the product.
    A general class diagram is provided and all external libraries are found here.
    \subsection{Project Structure}
    LibGDX enforces a certain project structure\cite{libgdxprojstruct}.
    The project root directory contains subdirectories for each different target system. (e.g. android, ios, desktop)
    These directories contain specifics for each different target and also acts as entry point for that target to our product.
    Furthermore there is the core directory which contains the actually source and business logic of our product.
    These directories contain specifics for each different target and also acts as entry point for that target to our product.
    All builds are done by predefined Gradle tasks.
    Gradle also manages the dependencies of the project.
    Next up is the documentation directory which, together with a local auxil directory make up our documentation, using \LaTeX\ of course.
    All project assets are stored inside the assets folder located inside the android folder
    This is because android builds require a certain project structure\cite{androidAssetLocation}.
    Finally there are a couple of build and property files together with the gitignore file.

    In the below diagram the core functionality of the entire Marama-View is described.\\
    This will be followed with an explanation of the different parts of this diagram for instance the LibGDX and the Screens part. \\
    \includegraphics[width=1\linewidth]{architecture-marama-view.png}

    \emph{Marama-View class diagram} \\
    \subsubsection[LibGDX]{LibGDX}
    The red part of the class diagram contains the LibGDX library.
    As can be seen, even though not every line is drawn (so the diagram is way clearer), LibGDX is used a lot.
    Many already existing classes are extended or used.
    This is done because on one side, we shouldn't write any code or build a structure that's already been thought of or worked out.
    On the other side the LibGDX library really fits our own perspective of what the structure of the application should be.
    Due to this, using the library happens pretty naturally and doesn't require much research.

    \subsubsection[MaramaViewLogic]{Marama-View logic}
    For a game, a few elements are needed. These elements can clearly be determined by looking at the class diagram. \\
    \begin{wrapfigure}{r}{5cm}
        \centering
        \includegraphics[width=0.4\textwidth, keepaspectratio]{marama-view-structure.jpg}
    \end{wrapfigure}
    The Marama-view is devided in multiple parts to enforce reusibillity and modularity.
    These parts are drawn in the image on the right.
    On the highest level, screens are defined.
    This is the current state of the game, and what the user can see.
    Examples of these screens are:
    \begin{itemize}
        \item The splash screen
        \item The main menu screen
        \item The in-game screen
    \end{itemize}
    A screen can contain multiple renderables that have to be rendered inside this screen.
    Renderables could be defined by layout parts, like a group of buttons, or a certain 3D area.
    Combining these parts one could create a functional view for the user.
    This way a screen can only contain one renderable as well.
    These renderables, mostly applicable to the 3D area, contain entities.
    Entities could be anything, and for clarification a block entity is drawn.
    All the blocks drawn actually exist out of one entity.
    Out of that entity, three instances are created.
    If you want a different shape, you should add an entity, and create an instance out of that.
    This system offers a lot of modularity since it is not complicated to add more entities, more screens and more renderables.
    It is a clear system that is very easy-to-use and easy-to-understand.
    \newpage

    %------------------------------------------------------------------------------------------------------------------------------------------------------- Design Choices
    \section{Design Choices}

    \subsection{Epics}
    The view is made up of different epics.
    Epics are a collection of user stories that share a goal or functionality.

    \newcommand{\clickup}[1]{https://app.clickup.com/757520/761304/t/#1}

    \subsubsection{3D Camera Controls}
    \myparagraph{\href{\clickup{2e2ca}}{\#2e2ca} As a MGD or player I want to zoom in/out on the world when pinching}
    The camera zoom functionality is already implemented and enabled by libGDX on all supported devices (Desktop, Android).
    \myparagraph{\href{\clickup{2e2c1}}{\#2e2c1} As a MGD or player I want to rotate the camera by one-fingered swiping}
    The camera rotation functionality is already implemented and enabled by libGDX on all supported devices (Desktop, Android).

    \subsubsection{UI}
    \myparagraph{\href{\clickup{302qb}}{\#302qb} As a user I want to see the Marama logo on start-up}
    When the program is started, independent of which built is used, the splashscreen is displayed first.
    The splashscreen displays the marama-logo and fades out and finally transitions into the mainmenu.
    The fade-out time and total duration are variable and can be changed by changing their respective constants in the SplashScreen.java file.
    It has been decided to resizes texture assets to dimensions in the power of two.
    This is to maintain backwards compatibility for OpenGL ES 1.0, to keep all functionality (some libGDX features are otherwise not supported) and it is a bit more memory efficient.\cite{libgdxpottex}

    \myparagraph{Early sketch of the interface}
    \includegraphics[]{marama-view-sketch.jpg}
%    \begin{wrapfigure}{R}{1cm}
%        \centering
%        \rule{3cm}{7cm}
%        \includegraphics[width=3cm, height=7cm]{marama-view-sketch.jpg}
%    \end{wrapfigure}
    This sketch describes two pairs of different interfaces for the editor.
    The first pair shows different approaches to 'a pallette of maramafication'.
    The first outline includes a scrollable list and the second outline includes a (non-scrollable) pallette which the user can fill with the maramafication he or she needs at the moment.
    The second pair displays several ways to interact with an M-block in the editor.
    The first outline shows a detailed right-hand sided menu with all options for the selected M-block.
    The second outline includes the same options but instead they appear in a pop-up box on the selected M-block.
    The interface design is inspired by a 3D sandbox app for android.\footnote[1]{https://play.google.com/store/apps/details?id=com.octagonstudio.assemblr}

    \subsubsection{3D Object Controls}
    \myparagraph{\href{\clickup{2e29e}}{\#2e29e} As a MGD or player I want to be able to select a 3D object}
    Eventually, the user of the editor needs functionality for manipulating 3D objects.
    This means moving, rotating and deleting 3D objects in and from the editor.
    Before any manipulation can occur, the user needs to be able to select 3D objects that are rendered inside the editor.
    A common way for selecting objects in a 3D world is to add some sort of click functionality.
    When the user clicks on a 3D object that is rendered in the editor, the 3D object will be marked as selected.
    The selected object must be properly highlighted so the user knows which object is selected.

    Before a 3D object is able to be selected, the object needs to implement a BoundingBox.
    A BoundingBox can be calculated from a rendered 3D Object and can be seen as a rectangular box.
    This box is able to determine whether it is intersected by a ray.
    Rays can be seen as lasers that can be fired from and to a certain location in the World.
    The ray will be cast from the origin of the camera by the user when he/she clicks on the screen.
    The first BoundingBox that is intersected by the ray will be the selected object.
    Casting the rays from the camera origin and locating the first intersected BoundingBox will prevent that 3D objects in the background will be selected.

    \myparagraph{\href{\clickup{3v5e5}}{\##2e2b8} As a MGD or player I want to move a 3D object}
    After an object has been selected and the move tool has been equipped three axes will surround the object in question.
    By pulling or pushing on an axis the object moves along the axis.
    One manipulate an axis by clicking with the left mouse button and dragging or by touch and dragging depending on your platform.
    To calculate the distance an object should move the previous click event coordinates and current event coordinates are projected to the 3D world.
    Then the difference between these two 3-vectors will be used to select the new position of the object based on the used axis.

    \myparagraph{\href{\clickup{26uy5}}{\#26uy5} As a MGD or MG I want to be able to add M to world from the list}
    The list shows all the possible M-blocks that can eventually be added to the world.
    Only one m-bock can be selected at a time in this list.
    The one that is selected will be the one that will be added to the world, if the add tool is selected.
    These M-blocks are already loaded as assets in the entityManager, so that makes it quite simple to add these M-blocks to the world.
    The world contains an array of all the instances that are currently rendered in the world.
    Adding a M-block is as easy as retrieving the M-block from the entityManager, creating a new instance and adding this instance to the array.
    The location where the M-block is placed is worked out in another user story: \href{\clickup{2e294}}{\#2e294}.

    \myparagraph{\href{\clickup{2e2bq}}{\#2e2bq} As a MGD or player I want to be able to delete a 3D object from the world.}
Object deletion is handled through an \code{InputController}, which libgdx uses to catch events from pointer behaviour.
When deleting an \code{SelectableInstance}, the user selects an \code{SelectableInstance} using a \code{BoundingBox} and a \code{Ray} on the press of a button. This process is explained above: \href{\clickup{2e29e}}{\#2e29e}.
When this occurs the \code{SelectableInstance} is turned slightly transparent to function as a preview.
While a \code{SelectableInstance} is selected for deletion, the camera is locked.
A second click will delete the \code{SelectableInstance} by removing it from the list of \code{SelectableInstance} in the world.
If the click hits another \code{SelectableInstance}, that \code{SelectableInstance} becomes selected instead.
    If it clicks elsewhere it deselects entirely.

    \myparagraph{\href{\clickup{2e294}}{\#2e294} As a MGD or MG I want the blocks to connect while I am dragging it within a certain range.}
Using the \code{BoundingBox} you can find which \code{SelectableInstance} you are close to fairly easily.
However, there is no functionality in libGDX for finding the face of a \code{SelectableInstance}.
To remedy this a new addition was made to the \code{SelectableInstance}: Joints.
Every \code{SelectableInstance} has a list of Joints which contain a \code{Vector3} which states the location of the joints of the \code{SelectableInstance} that can be interacted with.
When the ray wants to interact with a \code{SelectableInstance}(See \href{\clickup{2e29e}}{\#2e29e}) it returns the location of the intersect point.
    The points get translated and compared to one another to find the closest face to the intersect point.
Then the new \code{SelectableInstance} can be added.
    The functionality is not wholly completed.
It uses the face of the target \code{SelectableInstance} and that of the origin \code{SelectableInstance} to position the new \code{SelectableInstance} in the correct spot.
    In the future it should also rotate so that the joints properly face one another.



    \newpage
    %------------------------------------------------------------------------------------------------------------------------------------------------------- Build Automation
    \section{Continous Intergration}
    In this section build automation and its constituent components will be discussed.
    How we applied it, the software used and the targets and platforms supported are among a few.
    \newpage

    %------------------------------------------------------------------------------------------------------------------------------------------------------- Continuing Development
    \section{Continuing Development}
    This section describes the plan on how to continue with development in the best way.
    The team worked on quite some user stories from April 9th, to June 15th.
    The following functionalities are most critical at the moment to start working on:
    \subsection{Communication from the View to the Editor}
    A lot of communication is necessary for the View and Editor to work together.
    The editor contains all marama logic that the View will visually represent.
    To represent anything that contains marama logic, the communication needs te be established.
    Research has already been done covering this topic an can be read at the \href{https://docs.google.com/document/d/1Y8Otrf0bv8S2L_Sy8z_-9-9obVQ05sHg25_af-n7i4U/edit}{devlog}.
    \subsection{Able to load in a 3D object received from the Editor}
    The Editor will determine the state that the view is in.
    This means that the visual 3D state in the View is somewhere saved in the Editor.
    The View will need to receive some information on which maramafications to show and how they look like.
    \subsection{Validating if objects can be connected trough Editor communication}
    This functionality needs some work in the Editor before this can be implemented in the View.
    Whenever a user wants to connect two maramafications together the Editor needs to check if this connection is valid.
    From the View, the Editor must be spoken with to decide whether the current move is valid.
    The View must react to valid or invalid responses and needs te make clear to the user why the is valid or invalid.
    \newpage
    %------------------------------------------------------------------------------------------------------------------------------------------------------- References
    \bibliography{references}
    \bibliographystyle{ieeetr} % options: apalike for apa, ieeetr for ieee
\end{document}
